\section{Graphs: An introduction}
\label{title}

\section{Some Problems}

\begin{itemize}
	\item Five princes - can we insist that the King’s 5 sons should have a region such that each pair have a common boundary (a point does not count). 
	\item Three houses, three utilities - can we draw a graph with no edges, crossing. \textit{This is $K_{3,3}$}
	\item Three friends problem -  certain three mutual friends or three mutual strangers. Looking for a coloured triangle: 
	\begin{itemize}
		\item $K_4$ - with coloured internals 
		\item $K_5$ - with coloured internals 
		\item $K_6$ - can we draw $K_6$ without?
		\end{itemize} 
	Pick a node: Of the 5 arcs leaving, three will have the same colour. 
	If any one of the edges connecting those vertices is the same, we have a triangle - but ditto if none 
	
\end{itemize}

Four colours, 7 bridges, n queens, knight’s tour --- All represented as graph problems 

\section{Some definitions}

A graph $G$ is a finite nonempty set $V$ of vertices, with a set $E$ of 2-elements subsets of $V$, known as edges. These are the vertex set and edge set of $G$. 

The number of vertices is the \textit{order} of $G$. The number of edges is its \textit{size}. (Typically these are denoted $n$ and $m$. )

Two vertices $u$ and $v$ are said to be \textit{adjacent} if there is an edge $uv$, otherwise they are nonadjacent. Similarly adjacent edges share an incident vertex, e.g. $uv$ and $uw$.

A loop is an arc which connects a vertex to itself. An indirect connection is one that involves multiple edges going via intermediate vertices. A \textit{simple} graph has no loops. 

\textit{Unlabelled} graphs - if there is no advantage to labelling the vertices. 

\textit{Nontrivial} graphs have order at least 2. \textit{Nonempty} graphs have size at least 1. 

The number of edges incident with a vertex is said to denote the order of that vertex. An \textit{isolated} vertex has degree 0. Degree 1 is an \textit{end-vertex}. 

\textbf{Theorem} In any graph, the sum of the degrees of the vertices is twice the number of edges. 

$v_1 + v_2 + \ldots + v_n = 2m$ 

$uv$ is the same as $vu$ - so each edge is counted twice. 

How many in a full graph? - assuming no loops 

\textbf{Corollary} Every graph has an even number of odd vertices. 

$2m$ must be even. 

\subsection{Adjacency Matrix}

\subsection{Regular graphs}

A regular graph is one where all local degrees are the same number. 

\begin{itemize}
	\item A $0-$regular graph is empty of arcs. 
	\item A $1-$regular is disconnected edges (i.e. that cannot exist). 	
	\item  A $2-$regular consists of one (or more) disconnected cycles --- more on cycles later. 
	\item $3-$regulars are called cubic graphs \footnote{e.g. \url{https://mathworld.wolfram.com/CubicGraph.html}} $\dfrac{m}{n} = \dfrac{3}{2}$ is necessary but not sufficient. 
\end{itemize}

\subsection{irregular Graphs}

An \textit{irregular} graph is one of order at least 2 where every vertex has different degree. 

\textbf{Theorem} irregular graphs cannot exist. 

(Pigeonhole principle)

An \textit{almost irregular} graph has exactly one pair of vertices with the same degree. 

\textit{Complementary} graphs $\overline{G}$ have the same vertex set and where $u$ and $v$ are adjacent in one if they are non-adjacent in the complement. 

Any node in $\overline{G}$ has order (n-1)- the companion node’s order 

degG-bar v  = n - 1 - Gbv

Theorem: for each integer n >=2 there are exactly two almost irregular graphs of order n and they are complements of each other. 


\section{AOB}

\begin{itemize}
	\item Complete - including the number of edges
	\item digraph
	\item Adjacency Matrix
	\item trail --- any collections of connected nodes 
	\item path --- a trail with no repeating nodes 
	\item closed trail --- starting and finishing in the same place
	\item A cycle --- a closed trail where only the initial and final nodes are the same 
	\item connected graph 
	\item Eulerian --- a closed trail containing every arc 
	\item A connected graph is Eulerian iff every node is odd --- Nontrivial \footnote{proof here \url{http://mathonline.wikidot.com/euler-s-theorem}}
	\item semi-Eulerian --- A connected graph is semi-Eulerian iff precisely two nodes have odd order  
\end{itemize}
