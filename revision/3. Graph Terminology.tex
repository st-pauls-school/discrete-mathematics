
\documentclass[a4paper,10pt]{exam}
\usepackage[inline]{enumitem} % for the inline enumerate list 
\usepackage{amsmath,amsfonts}
\usepackage{multicol}
\usepackage{tikz}
\usepackage{gensymb} % for the degree symbol
\usepackage{setspace}
\usepackage{url}
\usepackage{array} % for the < cmd and the fixed width tabular 

\def\papertitle{Graph Skills}
\def\attribution{\texttt{cah@stpaulsschool.org.uk} \\ SPS Maths} 

\title{\papertitle}

\date{February}

\printanswers

\begin{document}

\maketitle

Note, this is graph only material - not networks, so this includes Eulerian, Hamiltonian and planar issues, but not Dijkstra, TSP. 

\begin{questions}
    \question How many arcs in: 
    \begin{parts}
        \part $K_6$ 
        \part $C_7$
        \part $K_{3,4}$
        \part $K_n$
        \part $C_n$
        \part $K_{m,n}$
        \part a MST on a graph with $n$ vertices. 
    \end{parts}
    \question Draw example graphs that are:
    \begin{parts}
        \part simple and connected
        \part simple but not connected 
        \part connected but not simple 
        \part neither simple nor connected 
    \end{parts}
    \question True or False: A minimal spanning tree is a trail.
    \begin{solution}
        false: Vertices may repeat, arcs cannot, so an MST \emph{might} be a trail, but does not have to be. 
    \end{solution}
    \question State Euler's formula in terms of a planar graph. 
    \question Prove this theorem\footnote{known as The Handshaking Theorem for Planar Graphs}: The sum of all the degrees of the faces/regions of a connected planar graph is equal to twice the number of arcs ($\Sigma degree(f) = 2e$). 
    \begin{solution}
        Case one, an arc is on a region that is a cycle: in which case it has an internal and an external face, inside and outside the face. 
        Case two, it is not on a cycle, in which case it has two exposures to the infinite region. 

        Since each arc borders two different faces or will border an infinite face twice, it must contribute 2 to the sum of the number of degree of the faces. 
    \end{solution}
    \question If $G$ is a planar simple graph, prove that $e \le 3v - 6$
    \begin{solution}
        By Euler's formula, $v - e + f = 2$, and so $f = e - v + 2$. 

        By the Handshaking Theorem, $\Sigma degree(f) = 2e$, and so $2e = \Sigma degree(f) \ge 3f$. 

        Substituting for $f$, $2e \ge 3(e - v + 2)$, and so $2e \ge 3e - 3v + 6$, and so $e \le 3v - 6$.
    \end{solution}
    \question Show that $K_{2,n}$ is planar, for any value of $n$.
    \begin{solution}
        $K_{2,n}$ is planar, as it can be drawn with the two vertices of degree 2 on the outside, and the $n$ vertices of degree 1 on the inside.
    \end{solution}
    \question State Kuratowksi's Theorem. 
    \begin{solution}
        A graph is planar if and only if (iff) it does not contain $K_5$ or $K_{3,3}$ as a subgraph. 
    \end{solution}
    \question Use Kuratowksi's theorem to prove that all complete graphs for $K_n$ where $n \ge 5$ are non-planar. 
    \begin{solution}
        A complete graph $K_n$ contains $K_5$ as a subgraph, and so is non-planar.
    \end{solution}
    \question Prove that a graph is Eulerian if and only if it is connected and every vertex has even degree.
    \begin{solution}
        If a graph is Eulerian, then it has an Eulerian circuit, and so every vertex has even degree. 

        If a graph is connected and every vertex has even degree, then it has an Eulerian circuit.
    \end{solution}
    \question State Ore's Theorem.
    \begin{solution}
        If $G$ is a simple graph with $n \ge 3$ vertices and for every pair of non-adjacent vertices, the sum of their degrees is greater than or equal to $n$, then $G$ is Hamiltonian.
    \end{solution}
    \question Prove Ore's Theorem.

        


    
\end{questions}

\end{document}
