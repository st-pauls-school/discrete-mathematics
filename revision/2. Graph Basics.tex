
\documentclass[a4paper,10pt]{exam}
\usepackage[inline]{enumitem} % for the inline enumerate list 
\usepackage{amsmath,amsfonts}
\usepackage{multicol}
\usepackage{tikz}
\usepackage{gensymb} % for the degree symbol
\usepackage{setspace}
\usepackage{url}
\usepackage{array} % for the < cmd and the fixed width tabular 

\def\papertitle{Preliminaries}
\def\attribution{\texttt{cah@stpaulsschool.org.uk} \\ SPS Maths} 

\title{\papertitle}

\date{February}

\printanswers

\begin{document}


\maketitle



\begin{questions} 
	\question If $G$ is a simple graph with $n$ vertices, what is the maximum number of edges that $G$ can have?
    \begin{solution}
        The maximum number of edges that a simple graph with $n$ vertices can have is $\frac{n(n-1)}{2}$.
    \end{solution}
    \question If $G$ is a simple graph with $n$ vertices, what is the minimum number of edges that $G$ can have?
    \begin{solution}
        The minimum number of edges that a simple graph with $n$ vertices can have is 0.
    \end{solution}
    \question Prove that for any graph $G$ of order at least 2, the degree sequence has at least two equal entries.
    \begin{solution}
        The degree sequence of a graph is the list of the degrees of the vertices in the graph. The degree of a vertex is the number of edges incident to it. 
        \begin{parts}
            \part The degree of a vertex is a non-negative integer. 
            \part The degree sequence of a graph is a list of non-negative integers. 
            \part If a graph has at least two vertices, then the degree sequence has at least two equal entries: pigeonhole principle.
        \end{parts}
    \end{solution}
    \question Prove that every connected graph contains at least one spanning tree. 
    \begin{solution}
        A connected graph is a graph in which there is a path between every pair of vertices. A spanning tree of a graph is a subgraph that is a tree and connects all the vertices together. 
        \begin{parts}
            \part A graph with $n$ vertices and $n-1$ edges is a tree. 
            \part A connected graph with $n$ vertices and $n-1$ edges is a tree. 
            \part A connected graph with $n$ vertices and $n-1$ edges contains a spanning tree. 
        \end{parts}
    \end{solution}
    \question Give the adjacency matrix for $K_n$. 
    \begin{solution}
        The adjacency matrix for $K_n$ is the $n \times n$ matrix with all entries equal to 1, except for the diagonal entries which are all 0.
    \end{solution}
    \question Give the adjacency matrix for $P_n$ (reminder: $P_n$ is the simple chain, path, of $n$ vertices.).
    \begin{solution}
        The adjacency matrix for $P_n$ is the $n \times n$ matrix with 1s on the diagonal and 1s on the off-diagonal entries that are adjacent.
    \end{solution}
    \question Give the adjacency matrix for $C_n$ (reminder: $C_n$ is the simple cycle, or ring, of $n$ vertices).
    \begin{solution}
        The adjacency matrix for $C_n$ is the $n \times n$ matrix with 1s on the diagonal and 1s on the off-diagonal entries that are adjacent. The entry in the top right corner is a 1.
    \end{solution}
    \question Give the adjacency matrix for $K_{m,n}$.
    \begin{solution}
        The adjacency matrix for $K_{m,n}$ is the $(m+n) \times (m+n)$ matrix with the first $m$ rows and columns corresponding to the vertices in the first part of the bipartite graph and the last $n$ rows and columns corresponding to the vertices in the second part of the bipartite graph. The entries are 1 if the vertices are adjacent and 0 otherwise.
    \end{solution}

\end{questions}

\subsection*{Further reading}

\end{document}