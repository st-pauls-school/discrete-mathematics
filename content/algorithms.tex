\section{Algorithms}
\label{title}
Algorithsm are a process or rules to follow. They should:
\begin{itemize}
	\item take an input, give an output. For example \ldots 
	\item be deterministic, i.e. for the same input the output will always be the same, cf functional programming  
	\item and finite, cf computability. 
\end{itemize}

They could be recursive, then they need a base case and recursive step. Typically the recursive step makes the problem smaller, converging on the base case. 

An algorithm could:
\begin{itemize}
	\item involve \textbf{heuristics}, finding decent solutions quickly, but not sure to be optimal 
	\item be greedy, so exploit strong local solutions, with no regard to the bigger picture, e.g. hill-climbing 
	\item ad-hoc, an impromptu strategy, unique to a particular problem, hard to generalise.  
\end{itemize}

\subsection{Flowcharts}

Shapes

\subsection{Orders of Growth}

Order of growth - worst case, proportional to some function of the input size 

Comparison is done in extreme cases, e.g. n log n vs n2 

Offline vs online problems 

Bubble sort, then shuttle 
Quick sort 

Packing : next fit, first fit, first fit decreasing, full bin 

Bin Packing - as a decision problem - is NP Complete 

Knapsack problem  - portfolio management, 

